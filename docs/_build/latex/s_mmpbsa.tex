%% Generated by Sphinx.
\def\sphinxdocclass{report}
\documentclass[letterpaper,10pt,english]{sphinxmanual}
\ifdefined\pdfpxdimen
   \let\sphinxpxdimen\pdfpxdimen\else\newdimen\sphinxpxdimen
\fi \sphinxpxdimen=.75bp\relax
\ifdefined\pdfimageresolution
    \pdfimageresolution= \numexpr \dimexpr1in\relax/\sphinxpxdimen\relax
\fi
%% let collapsible pdf bookmarks panel have high depth per default
\PassOptionsToPackage{bookmarksdepth=5}{hyperref}
%% turn off hyperref patch of \index as sphinx.xdy xindy module takes care of
%% suitable \hyperpage mark-up, working around hyperref-xindy incompatibility
\PassOptionsToPackage{hyperindex=false}{hyperref}
%% memoir class requires extra handling
\makeatletter\@ifclassloaded{memoir}
{\ifdefined\memhyperindexfalse\memhyperindexfalse\fi}{}\makeatother

\PassOptionsToPackage{booktabs}{sphinx}
\PassOptionsToPackage{colorrows}{sphinx}

\PassOptionsToPackage{warn}{textcomp}

\catcode`^^^^00a0\active\protected\def^^^^00a0{\leavevmode\nobreak\ }
\usepackage{cmap}
\usepackage{xeCJK}
\usepackage{amsmath,amssymb,amstext}
\usepackage{babel}



\setmainfont{FreeSerif}[
  Extension      = .otf,
  UprightFont    = *,
  ItalicFont     = *Italic,
  BoldFont       = *Bold,
  BoldItalicFont = *BoldItalic
]
\setsansfont{FreeSans}[
  Extension      = .otf,
  UprightFont    = *,
  ItalicFont     = *Oblique,
  BoldFont       = *Bold,
  BoldItalicFont = *BoldOblique,
]
\setmonofont{FreeMono}[Scale=0.9,
  Extension      = .otf,
  UprightFont    = *,
  ItalicFont     = *Oblique,
  BoldFont       = *Bold,
  BoldItalicFont = *BoldOblique,
]



\usepackage[Sonny]{fncychap}
\ChNameVar{\Large\normalfont\sffamily}
\ChTitleVar{\Large\normalfont\sffamily}
\usepackage{sphinx}

\fvset{fontsize=\small,formatcom=\xeCJKVerbAddon}
\usepackage{geometry}


% Include hyperref last.
\usepackage{hyperref}
% Fix anchor placement for figures with captions.
\usepackage{hypcap}% it must be loaded after hyperref.
% Set up styles of URL: it should be placed after hyperref.
\urlstyle{same}

\addto\captionsenglish{\renewcommand{\contentsname}{内容概述}}

\usepackage{sphinxmessages}
\setcounter{tocdepth}{1}



\title{s\_mmpbsa}
\date{2025 年 09 月 01 日}
\release{1.0.0}
\author{Supernova}
\newcommand{\sphinxlogo}{\vbox{}}
\renewcommand{\releasename}{发行版本}
\makeindex
\begin{document}

\ifdefined\shorthandoff
  \ifnum\catcode`\=\string=\active\shorthandoff{=}\fi
  \ifnum\catcode`\"=\active\shorthandoff{"}\fi
\fi

\pagestyle{empty}
\sphinxmaketitle
\pagestyle{plain}
\sphinxtableofcontents
\pagestyle{normal}
\phantomsection\label{\detokenize{index::doc}}


\sphinxAtStartPar
\sphinxstylestrong{s\_mmpbsa} 是一个用于计算生物分子结合自由能的高效工具,专门用于分析Gromacs轨迹,采用分子力学泊松\sphinxhyphen{}玻尔兹曼表面积(MM/PB\sphinxhyphen{}SA)方法。

\sphinxstepscope


\chapter{介绍}
\label{\detokenize{introduction:id1}}\label{\detokenize{introduction::doc}}
\sphinxAtStartPar
本文档介绍s\_mmpbsa的背景、原理和基本概念,帮助用户理解该工具的工作原理和应用场景。


\section{MM/PB\sphinxhyphen{}SA方法简介}
\label{\detokenize{introduction:mm-pb-sa}}
\sphinxAtStartPar
{\color{red}\bfseries{}**}MM/PB\sphinxhyphen{}SA**(Molecular Mechanics/Poisson\sphinxhyphen{}Boltzmann Surface Area)是一种广泛用于计算生物分子结合自由能的方法。该方法结合了分子力学(MM)和连续溶剂模型(PB\sphinxhyphen{}SA),能够快速准确地预测生物分子间的相互作用强度。

\sphinxAtStartPar
MM/PB\sphinxhyphen{}SA方法的基本原理是通过计算结合前后的自由能变化来评估分子间的结合强度。具体而言,结合自由能(ΔG)可以表示为:
\begin{equation*}
\begin{split}\Delta G_{binding} = \Delta G_{complex} - (\Delta G_{receptor} + \Delta G_{ligand})\end{split}
\end{equation*}
\sphinxAtStartPar
其中,每个分子的自由能(G)由以下几个部分组成:
\begin{equation*}
\begin{split}G = E_{MM} + G_{solv} - T\Delta S\end{split}
\end{equation*}\begin{itemize}
\item {} 
\sphinxAtStartPar
\sphinxstylestrong{E\_\{MM\}}:分子力学能量,包括键能、角度能、二面角能、范德华能和静电能

\item {} 
\sphinxAtStartPar
\sphinxstylestrong{G\_\{solv\}}:溶剂化自由能,包括极性溶剂化能(通过泊松\sphinxhyphen{}玻尔兹曼方程计算)和非极性溶剂化能(通过表面积计算)

\item {} 
\sphinxAtStartPar
\sphinxstylestrong{TΔS}:熵贡献项,通常通过正常模式分析计算

\end{itemize}


\section{为什么选择s\_mmpbsa?}
\label{\detokenize{introduction:s-mmpbsa}}
\sphinxAtStartPar
尽管Gromacs是一种广泛使用的分子动力学模拟软件,但它并没有官方支持MM/PB\sphinxhyphen{}SA计算。目前市场上有许多可以处理Gromacs轨迹的MM/PB\sphinxhyphen{}SA工具,但它们大多存在以下局限性:
\begin{enumerate}
\sphinxsetlistlabels{\arabic}{enumi}{enumii}{}{.}%
\item {} 
\sphinxAtStartPar
\sphinxstylestrong{安装和使用复杂}

\item {} 
\sphinxAtStartPar
\sphinxstylestrong{不支持新版本的Gromacs}

\item {} 
\sphinxAtStartPar
\sphinxstylestrong{计算速度慢}

\item {} 
\sphinxAtStartPar
\sphinxstylestrong{不支持跨平台}

\end{enumerate}

\sphinxAtStartPar
相比之下,s\_mmpbsa提供了以下优势:
\begin{itemize}
\item {} 
\sphinxAtStartPar
\sphinxstylestrong{简单易用}:交互式操作界面,无需编写复杂的参数文件

\item {} 
\sphinxAtStartPar
\sphinxstylestrong{高效计算}:使用Rust语言开发,计算速度快

\item {} 
\sphinxAtStartPar
\sphinxstylestrong{跨平台}:支持Windows和Linux操作系统

\item {} 
\sphinxAtStartPar
\sphinxstylestrong{功能丰富}:支持电荷筛选效应和构象熵计算

\item {} 
\sphinxAtStartPar
\sphinxstylestrong{易于集成}:可通过脚本调用,支持批量处理

\end{itemize}


\section{s\_mmpbsa的基本工作流程}
\label{\detokenize{introduction:id4}}
\sphinxAtStartPar
s\_mmpbsa的工作流程主要包括以下几个步骤:
\begin{enumerate}
\sphinxsetlistlabels{\arabic}{enumi}{enumii}{}{.}%
\item {} 
\sphinxAtStartPar
\sphinxstylestrong{输入处理}:读取Gromacs的tpr、xtc和ndx文件

\item {} 
\sphinxAtStartPar
\sphinxstylestrong{轨迹处理}:处理分子动力学轨迹,包括提取坐标和处理周期性边界条件

\item {} 
\sphinxAtStartPar
\sphinxstylestrong{MM计算}:计算分子力学能量(键能、范德华能和静电能等)

\item {} 
\sphinxAtStartPar
\sphinxstylestrong{PB\sphinxhyphen{}SA计算}:计算溶剂化自由能(极性和非极性)

\item {} 
\sphinxAtStartPar
\sphinxstylestrong{熵计算}:计算构象熵贡献(可选)

\item {} 
\sphinxAtStartPar
\sphinxstylestrong{结果分析}:生成结合自由能报告和可视化结果

\end{enumerate}


\section{应用场景}
\label{\detokenize{introduction:id5}}
\sphinxAtStartPar
s\_mmpbsa适用于以下研究场景:
\begin{enumerate}
\sphinxsetlistlabels{\arabic}{enumi}{enumii}{}{.}%
\item {} 
\sphinxAtStartPar
\sphinxstylestrong{药物设计}:评估药物分子与靶点的结合强度,指导药物优化

\item {} 
\sphinxAtStartPar
\sphinxstylestrong{蛋白质\sphinxhyphen{}蛋白质相互作用}:研究蛋白质复合物的稳定性和相互作用机制

\item {} 
\sphinxAtStartPar
\sphinxstylestrong{酶\sphinxhyphen{}底物相互作用}:分析酶催化反应中的结合自由能变化

\item {} 
\sphinxAtStartPar
\sphinxstylestrong{突变效应预测}:通过丙氨酸扫描评估蛋白质中关键残基对结合的贡献

\item {} 
\sphinxAtStartPar
\sphinxstylestrong{分子对接结果验证}:为分子对接结果提供更准确的结合能预测

\end{enumerate}


\section{理论创新点}
\label{\detokenize{introduction:id6}}
\sphinxAtStartPar
s\_mmpbsa在传统MM/PB\sphinxhyphen{}SA方法的基础上进行了以下改进:
\begin{enumerate}
\sphinxsetlistlabels{\arabic}{enumi}{enumii}{}{.}%
\item {} 
\sphinxAtStartPar
\sphinxstylestrong{电荷筛选效应}:考虑了生物分子环境中的电荷筛选效应,提高了极性相互作用计算的准确性(参考文献:J. Chem. Inf. Model. 2021, 61, 2454)

\item {} 
\sphinxAtStartPar
\sphinxstylestrong{构象熵计算}:实现了高效的构象熵计算方法,为结合自由能预测提供更全面的热力学信息(参考文献:J. Chem. Phys. 2017, 146, 124124)

\item {} 
\sphinxAtStartPar
\sphinxstylestrong{并行计算优化}:通过Rust语言的并行特性,显著提高了计算效率,特别是对于大型生物分子系统

\item {} 
\sphinxAtStartPar
\sphinxstylestrong{结果可视化}:提供了丰富的结果分析和可视化功能,便于用户理解和解释计算结果

\end{enumerate}


\section{许可证信息}
\label{\detokenize{introduction:id7}}
\sphinxAtStartPar
s\_mmpbsa遵循LGPL许可证,可以免费用于学术研究目的。如果您在商业环境中使用s\_mmpbsa,请确保遵守LGPL许可证的相关规定。


\section{引用s\_mmpbsa}
\label{\detokenize{introduction:id8}}
\sphinxAtStartPar
如果您在研究工作中使用了s\_mmpbsa,请按照以下格式引用:

\begin{sphinxVerbatim}[commandchars=\\\{\}]
Jiaxing Zhang, s\PYGZus{}mmpbsa, Version [your version], https://github.com/supernova4869/s\PYGZus{}mmpbsa (accessed on yy\PYGZhy{}mm\PYGZhy{}dd)
\end{sphinxVerbatim}

\sphinxAtStartPar
我们正在准备关于s\_mmpbsa的详细学术论文,发表后请引用相应的论文。

\sphinxstepscope


\chapter{安装}
\label{\detokenize{installation:id1}}\label{\detokenize{installation::doc}}
\sphinxAtStartPar
本文档介绍s\_mmpbsa的安装步骤、环境要求和配置方法,帮助用户快速搭建运行环境。


\section{系统要求}
\label{\detokenize{installation:id2}}
\sphinxAtStartPar
\#\#\# 操作系统

\sphinxAtStartPar
s\_mmpbsa支持以下操作系统:
\begin{itemize}
\item {} 
\sphinxAtStartPar
\sphinxstylestrong{Windows}:Windows 7/8/10/11

\item {} 
\sphinxAtStartPar
\sphinxstylestrong{Linux}:Ubuntu、Debian、CentOS、Rocky等主流Linux发行版

\end{itemize}

\sphinxAtStartPar
\#\#\# 硬件要求
\begin{itemize}
\item {} 
\sphinxAtStartPar
\sphinxstylestrong{处理器}:多核处理器(推荐4核及以上)

\item {} 
\sphinxAtStartPar
\sphinxstylestrong{内存}:至少4GB RAM(大型系统推荐8GB及以上)

\item {} 
\sphinxAtStartPar
\sphinxstylestrong{磁盘空间}:至少500MB可用空间

\end{itemize}


\section{软件依赖}
\label{\detokenize{installation:id3}}
\sphinxAtStartPar
\#\#\# 基本依赖

\sphinxAtStartPar
s\_mmpbsa的核心功能需要以下软件:
\begin{itemize}
\item {} 
\sphinxAtStartPar
\sphinxstylestrong{Gromacs}:用于处理分子动力学轨迹文件。s\_mmpbsa支持多个版本的Gromacs,但建议使用较新版本以获得最佳兼容性。

\end{itemize}

\sphinxAtStartPar
可选依赖:
\begin{itemize}
\item {} 
\sphinxAtStartPar
\sphinxstylestrong{Matplotlib}:用于绘制结果图表。如果您需要使用s\_mmpbsa的分析和绘图功能,则需要安装。

\item {} 
\sphinxAtStartPar
\sphinxstylestrong{APBS}:用于计算泊松\sphinxhyphen{}玻尔兹曼表面面积。s\_mmpbsa内置了APBS内核,但也支持使用外部APBS程序。

\end{itemize}

\sphinxAtStartPar
\#\#\# 分子对接重打分功能的特殊依赖
\begin{itemize}
\item {} 
\sphinxAtStartPar
\sphinxstylestrong{PyMOL}:可选软件,用于绘制B因子着色的结构。

\item {} 
\sphinxAtStartPar
\sphinxstylestrong{Gaussian}:可选软件,用于RESP原子电荷计算的DFT计算。

\item {} 
\sphinxAtStartPar
\sphinxstylestrong{Multiwfn}:可选程序(已内置),用于拟合RESP原子电荷。

\end{itemize}


\section{安装步骤}
\label{\detokenize{installation:id4}}
\sphinxAtStartPar
\#\#\# Windows系统安装
\begin{enumerate}
\sphinxsetlistlabels{\arabic}{enumi}{enumii}{}{.}%
\item {} 
\sphinxAtStartPar
\sphinxstylestrong{下载s\_mmpbsa}

\sphinxAtStartPar
从GitHub发布页面下载最新版本的Windows可执行文件:

\begin{sphinxVerbatim}[commandchars=\\\{\}]
\PYG{c}{\PYGZsh{} 从GitHub下载s\PYGZus{}mmpbsa.exe}
\PYG{c}{\PYGZsh{} 访问: https://github.com/supernova4869/s\PYGZus{}mmpbsa/releases/latest}
\end{sphinxVerbatim}

\item {} 
\sphinxAtStartPar
\sphinxstylestrong{添加到系统路径}

\sphinxAtStartPar
将s\_mmpbsa.exe所在的文件夹添加到系统环境变量PATH中,以便在任何位置都能运行s\_mmpbsa。

\item {} 
\sphinxAtStartPar
\sphinxstylestrong{安装Gromacs}

\sphinxAtStartPar
从Gromacs官方网站下载并安装适合您系统的Gromacs版本。

\item {} 
\sphinxAtStartPar
\sphinxstylestrong{安装可选依赖(如需使用分析功能)}

\begin{sphinxVerbatim}[commandchars=\\\{\}]
\PYG{c}{\PYGZsh{} 安装matplotlib}
\PYG{n}{pip} \PYG{n}{install} \PYG{n}{matplotlib}
\end{sphinxVerbatim}

\end{enumerate}

\sphinxAtStartPar
\#\#\# Linux系统安装

\sphinxAtStartPar
\#\#\#\# Ubuntu/Debian系统
\begin{enumerate}
\sphinxsetlistlabels{\arabic}{enumi}{enumii}{}{.}%
\item {} 
\sphinxAtStartPar
\sphinxstylestrong{下载s\_mmpbsa}

\begin{sphinxVerbatim}[commandchars=\\\{\}]
\PYG{c+c1}{\PYGZsh{} 从GitHub下载最新版本}
wget\PYG{+w}{ }https://github.com/supernova4869/s\PYGZus{}mmpbsa/releases/latest/download/s\PYGZus{}mmpbsa

\PYG{c+c1}{\PYGZsh{} 添加执行权限}
chmod\PYG{+w}{ }+x\PYG{+w}{ }s\PYGZus{}mmpbsa
\end{sphinxVerbatim}

\item {} 
\sphinxAtStartPar
\sphinxstylestrong{添加到系统路径}

\begin{sphinxVerbatim}[commandchars=\\\{\}]
\PYG{c+c1}{\PYGZsh{} 将s\PYGZus{}mmpbsa移动到/usr/local/bin或其他已在PATH中的目录}
sudo\PYG{+w}{ }mv\PYG{+w}{ }s\PYGZus{}mmpbsa\PYG{+w}{ }/usr/local/bin/
\end{sphinxVerbatim}

\item {} 
\sphinxAtStartPar
\sphinxstylestrong{安装必要依赖}

\begin{sphinxVerbatim}[commandchars=\\\{\}]
\PYG{c+c1}{\PYGZsh{} 安装Gromacs(如果尚未安装)}
sudo\PYG{+w}{ }apt\PYGZhy{}get\PYG{+w}{ }update
sudo\PYG{+w}{ }apt\PYGZhy{}get\PYG{+w}{ }install\PYG{+w}{ }gromacs

\PYG{c+c1}{\PYGZsh{} 安装matplotlib和其他必要的Python包}
sudo\PYG{+w}{ }apt\PYG{+w}{ }\PYGZhy{}y\PYG{+w}{ }install\PYG{+w}{ }python3\PYGZhy{}matplotlib\PYG{+w}{ }build\PYGZhy{}essential\PYG{+w}{ }python\PYGZhy{}pip
\end{sphinxVerbatim}

\end{enumerate}

\sphinxAtStartPar
\#\#\#\# CentOS/Rocky系统
\begin{enumerate}
\sphinxsetlistlabels{\arabic}{enumi}{enumii}{}{.}%
\item {} 
\sphinxAtStartPar
\sphinxstylestrong{下载s\_mmpbsa}

\begin{sphinxVerbatim}[commandchars=\\\{\}]
\PYG{c+c1}{\PYGZsh{} 从GitHub下载最新版本}
wget\PYG{+w}{ }https://github.com/supernova4869/s\PYGZus{}mmpbsa/releases/latest/download/s\PYGZus{}mmpbsa

\PYG{c+c1}{\PYGZsh{} 添加执行权限}
chmod\PYG{+w}{ }+x\PYG{+w}{ }s\PYGZus{}mmpbsa
\end{sphinxVerbatim}

\item {} 
\sphinxAtStartPar
\sphinxstylestrong{添加到系统路径}

\begin{sphinxVerbatim}[commandchars=\\\{\}]
\PYG{c+c1}{\PYGZsh{} 将s\PYGZus{}mmpbsa移动到/usr/local/bin或其他已在PATH中的目录}
sudo\PYG{+w}{ }mv\PYG{+w}{ }s\PYGZus{}mmpbsa\PYG{+w}{ }/usr/local/bin/
\end{sphinxVerbatim}

\item {} 
\sphinxAtStartPar
\sphinxstylestrong{安装必要依赖}

\begin{sphinxVerbatim}[commandchars=\\\{\}]
\PYG{c+c1}{\PYGZsh{} 安装Gromacs(可能需要从源码编译或使用EPEL仓库)}
\PYG{c+c1}{\PYGZsh{} 这里假设您已经安装了Gromacs}

\PYG{c+c1}{\PYGZsh{} 安装matplotlib和其他必要的Python包}
sudo\PYG{+w}{ }dnf\PYG{+w}{ }\PYGZhy{}y\PYG{+w}{ }install\PYG{+w}{ }python3\PYGZhy{}matplotlib\PYG{+w}{ }python\PYGZhy{}pip
\end{sphinxVerbatim}

\end{enumerate}


\section{验证安装}
\label{\detokenize{installation:id5}}
\sphinxAtStartPar
安装完成后,可以通过以下方式验证s\_mmpbsa是否正确安装:

\begin{sphinxVerbatim}[commandchars=\\\{\}]
\PYG{c+c1}{\PYGZsh{} 在命令行中运行}
s\PYGZus{}mmpbsa\PYG{+w}{ }\PYGZhy{}\PYGZhy{}version

\PYG{c+c1}{\PYGZsh{} 或者直接运行s\PYGZus{}mmpbsa}
s\PYGZus{}mmpbsa
\end{sphinxVerbatim}

\sphinxAtStartPar
如果安装成功,您将看到s\_mmpbsa的欢迎信息和版本号。


\section{配置s\_mmpbsa}
\label{\detokenize{installation:s-mmpbsa}}
\sphinxAtStartPar
s\_mmpbsa的配置文件为`settings.ini`,该文件包含了程序的各种设置参数。您可以根据需要修改这些参数以优化程序性能或调整计算设置。

\sphinxAtStartPar
\#\#\# 配置文件位置
\begin{itemize}
\item {} 
\sphinxAtStartPar
Windows系统:配置文件通常位于s\_mmpbsa可执行文件所在的目录

\item {} 
\sphinxAtStartPar
Linux系统:配置文件通常位于`\textasciitilde{}/.s\_mmpbsa/{\color{red}\bfseries{}`}目录

\end{itemize}

\sphinxAtStartPar
\#\#\# 主要配置参数

\sphinxAtStartPar
配置文件中包含以下主要参数:
\begin{itemize}
\item {} 
\sphinxAtStartPar
\sphinxstylestrong{gmx\_path}:Gromacs程序的路径

\item {} 
\sphinxAtStartPar
\sphinxstylestrong{apbs\_path}:APBS程序的路径(如果使用外部APBS)

\item {} 
\sphinxAtStartPar
\sphinxstylestrong{nkernels}:并行计算使用的核心数

\item {} 
\sphinxAtStartPar
\sphinxstylestrong{debug\_mode}:是否启用调试模式

\item {} 
\sphinxAtStartPar
\sphinxstylestrong{r\_cutoff}:非键相互作用的截断距离

\item {} 
\sphinxAtStartPar
\sphinxstylestrong{elec\_screen}:电荷筛选方法设置

\end{itemize}

\sphinxAtStartPar
\#\#\# 配置文件示例

\sphinxAtStartPar
{\color{red}\bfseries{}``}{\color{red}\bfseries{}`}ini
{[}General{]}
last\_opened = ""
debug\_mode = false
nkernels = 4

\sphinxAtStartPar
{[}Program{]}
gmx\_path = "built\sphinxhyphen{}in"
apbs\_path = "built\sphinxhyphen{}in"
delphi\_path = ""

\sphinxAtStartPar
{[}Parameters{]}
r\_cutoff = 1.2
elec\_screen = 0

\sphinxAtStartPar
{[}Display{]}
font\_size = 12
{\color{red}\bfseries{}``}{\color{red}\bfseries{}`}


\section{常见安装问题}
\label{\detokenize{installation:id16}}
\sphinxAtStartPar
\#\#\# Gromacs未找到

\sphinxAtStartPar
如果s\_mmpbsa无法找到Gromacs程序,请确保:
\begin{enumerate}
\sphinxsetlistlabels{\arabic}{enumi}{enumii}{}{.}%
\item {} 
\sphinxAtStartPar
Gromacs已正确安装

\item {} 
\sphinxAtStartPar
Gromacs的可执行文件所在目录已添加到系统环境变量PATH中

\item {} 
\sphinxAtStartPar
在settings.ini中正确设置了gmx\_path参数

\end{enumerate}

\sphinxAtStartPar
\#\#\# APBS相关错误

\sphinxAtStartPar
如果使用内置APBS内核出现问题,可以尝试:
\begin{enumerate}
\sphinxsetlistlabels{\arabic}{enumi}{enumii}{}{.}%
\item {} 
\sphinxAtStartPar
安装外部APBS程序

\item {} 
\sphinxAtStartPar
在settings.ini中设置apbs\_path参数指向外部APBS程序

\end{enumerate}

\sphinxAtStartPar
\#\#\# Python/matplotlib相关错误

\sphinxAtStartPar
如果在使用分析功能时出现Python或matplotlib相关错误,请确保:
\begin{enumerate}
\sphinxsetlistlabels{\arabic}{enumi}{enumii}{}{.}%
\item {} 
\sphinxAtStartPar
已安装正确版本的Python(推荐Python 3.6及以上)

\item {} 
\sphinxAtStartPar
已安装matplotlib包:\sphinxtitleref{pip install matplotlib}

\end{enumerate}

\sphinxAtStartPar
\#\#\# 性能问题

\sphinxAtStartPar
如果计算速度较慢,可以尝试:
\begin{enumerate}
\sphinxsetlistlabels{\arabic}{enumi}{enumii}{}{.}%
\item {} 
\sphinxAtStartPar
在settings.ini中增加nkernels参数的值,利用更多CPU核心

\item {} 
\sphinxAtStartPar
对于大型系统,考虑增加计算的时间间隔(即减少分析的帧数)

\end{enumerate}


\section{获取帮助}
\label{\detokenize{installation:id17}}
\sphinxAtStartPar
如果您在安装过程中遇到任何问题,可以:
\begin{itemize}
\item {} 
\sphinxAtStartPar
查看GitHub仓库中的问题页面:\sphinxurl{https://github.com/supernova4869/s\_mmpbsa/issues}

\item {} 
\sphinxAtStartPar
联系开发者:\sphinxhref{mailto:zhangjiaxing7137@tju.edu.cn}{zhangjiaxing7137@tju.edu.cn}

\item {} 
\sphinxAtStartPar
加入QQ群:864191465

\end{itemize}

\sphinxstepscope


\chapter{快速入门}
\label{\detokenize{quick_start:id1}}\label{\detokenize{quick_start::doc}}
\sphinxAtStartPar
本文档提供s\_mmpbsa的基本使用流程,帮助您快速上手该工具进行结合自由能计算。


\section{启动s\_mmpbsa}
\label{\detokenize{quick_start:s-mmpbsa}}
\sphinxAtStartPar
安装完成后,可以通过以下方式启动s\_mmpbsa:

\begin{sphinxVerbatim}[commandchars=\\\{\}]
\PYG{c+c1}{\PYGZsh{} 在命令行中直接运行}
s\PYGZus{}mmpbsa

\PYG{c+c1}{\PYGZsh{} 或者直接指定tpr文件路径}
s\PYGZus{}mmpbsa\PYG{+w}{ }md.tpr
\end{sphinxVerbatim}

\sphinxAtStartPar
启动后,您将看到s\_mmpbsa的欢迎信息,然后进入交互式界面。


\section{基本工作流程}
\label{\detokenize{quick_start:id2}}
\sphinxAtStartPar
s\_mmpbsa的基本工作流程包括以下几个步骤:
\begin{enumerate}
\sphinxsetlistlabels{\arabic}{enumi}{enumii}{}{.}%
\item {} 
\sphinxAtStartPar
加载输入文件(tpr、xtc和ndx文件)

\item {} 
\sphinxAtStartPar
设置轨迹参数(选择受体和配体组)

\item {} 
\sphinxAtStartPar
设置MM/PB\sphinxhyphen{}SA参数

\item {} 
\sphinxAtStartPar
执行计算

\item {} 
\sphinxAtStartPar
分析结果

\end{enumerate}

\sphinxAtStartPar
下面我们将详细介绍每个步骤的操作方法。


\section{加载输入文件}
\label{\detokenize{quick_start:id3}}
\sphinxAtStartPar
启动s\_mmpbsa后,首先需要加载必要的输入文件:

\begin{sphinxVerbatim}[commandchars=\\\{\}]
\PYG{c+c1}{\PYGZsh{} 输入tpr文件路径}
md.tpr

\PYG{c+c1}{\PYGZsh{} 加载xtc文件(选项1)}
\PYG{l+m}{1}
md\PYGZus{}centered.xtc\PYG{+w}{  }\PYG{c+c1}{\PYGZsh{} 如果轨迹已处理过PBC,可以直接输入;否则按回车使用默认的md.xtc}

\PYG{c+c1}{\PYGZsh{} 加载ndx文件(选项2)}
\PYG{l+m}{2}
index.ndx\PYG{+w}{  }\PYG{c+c1}{\PYGZsh{} 按回车使用默认的index.ndx}

\PYG{c+c1}{\PYGZsh{} 进入下一步(选项0)}
\PYG{l+m}{0}
\end{sphinxVerbatim}


\section{设置轨迹参数}
\label{\detokenize{quick_start:id4}}
\sphinxAtStartPar
加载输入文件后,需要设置轨迹参数,主要是选择受体和配体组:

\begin{sphinxVerbatim}[commandchars=\\\{\}]
\PYG{c+c1}{\PYGZsh{} 选择受体组(选项1)}
\PYG{l+m}{1}
\PYG{o}{[}选择受体组的编号,例如1表示Protein\PYG{o}{]}

\PYG{c+c1}{\PYGZsh{} 选择配体组(选项2)}
\PYG{l+m}{2}
\PYG{o}{[}选择配体组的编号,例如13表示配体\PYG{o}{]}

\PYG{c+c1}{\PYGZsh{} 设置时间间隔(选项5),通常每1ns分析一次}
\PYG{l+m}{5}
\PYG{l+m}{1}\PYG{+w}{  }\PYG{c+c1}{\PYGZsh{} 时间间隔,单位为ns}

\PYG{c+c1}{\PYGZsh{} 进入下一步(选项0)}
\PYG{l+m}{0}
\end{sphinxVerbatim}


\section{设置MM/PB\sphinxhyphen{}SA参数}
\label{\detokenize{quick_start:mm-pb-sa}}
\sphinxAtStartPar
接下来,设置MM/PB\sphinxhyphen{}SA计算的相关参数:

\begin{sphinxVerbatim}[commandchars=\\\{\}]
\PYG{c+c1}{\PYGZsh{} 通常情况下,使用默认参数即可}
\PYG{c+c1}{\PYGZsh{} 如需修改PB参数,可以选择选项8}
\PYG{c+c1}{\PYGZsh{} 如需修改SA参数,可以选择选项9}

\PYG{c+c1}{\PYGZsh{} 进入下一步(选项0)}
\PYG{l+m}{0}
\end{sphinxVerbatim}


\section{执行计算}
\label{\detokenize{quick_start:id5}}
\sphinxAtStartPar
设置完成后,开始执行计算:

\begin{sphinxVerbatim}[commandchars=\\\{\}]
\PYG{c+c1}{\PYGZsh{} 输入系统名称(默认为system)}
\PYG{o}{[}按回车使用默认名称或输入自定义名称\PYG{o}{]}

\PYG{c+c1}{\PYGZsh{} 等待计算完成}
\PYG{c+c1}{\PYGZsh{} 计算过程中会显示进度条和当前能量值}
\end{sphinxVerbatim}


\section{分析结果}
\label{\detokenize{quick_start:id6}}
\sphinxAtStartPar
计算完成后,可以进行结果分析:

\begin{sphinxVerbatim}[commandchars=\\\{\}]
\PYG{c+c1}{\PYGZsh{} 生成包含残基结合能信息的pdb文件(选项\PYGZhy{}1)}
\PYGZhy{}1
\PYG{o}{[}按回车使用默认时间点(平均值)或输入特定时间点\PYG{o}{]}

\PYG{c+c1}{\PYGZsh{} 查看结果摘要(选项1)}
\PYG{l+m}{1}

\PYG{c+c1}{\PYGZsh{} 输出能量随时间变化的数据(选项2)}
\PYG{l+m}{2}

\PYG{c+c1}{\PYGZsh{} 输出特定时间点的残基结合能(选项3)}
\PYG{l+m}{3}
\PYG{o}{[}按回车使用默认时间点(平均值)或输入特定时间点\PYG{o}{]}
\PYG{l+m}{1}\PYG{+w}{  }\PYG{c+c1}{\PYGZsh{} 选择输出3Å范围内的残基}

\PYG{c+c1}{\PYGZsh{} 输出配体原子的结合能(选项4)}
\PYG{l+m}{4}

\PYG{c+c1}{\PYGZsh{} 退出程序(选项0)}
\PYG{l+m}{0}
\end{sphinxVerbatim}


\section{使用分析模式}
\label{\detokenize{quick_start:id7}}
\sphinxAtStartPar
s\_mmpbsa还提供了专门的分析模式,可以对已计算的结果进行重新分析:

\begin{sphinxVerbatim}[commandchars=\\\{\}]
\PYG{c+c1}{\PYGZsh{} 启动分析模式}
s\PYGZus{}mmpbsa
a\PYG{+w}{  }\PYG{c+c1}{\PYGZsh{} 输入a进入分析模式}

\PYG{c+c1}{\PYGZsh{} 输入工作目录路径(默认为当前目录)}
\PYG{o}{[}按回车使用当前目录或输入包含.sm结果文件的目录\PYG{o}{]}

\PYG{c+c1}{\PYGZsh{} 输入温度(默认为298.15K)}
\PYG{o}{[}按回车使用默认温度或输入自定义温度\PYG{o}{]}

\PYG{c+c1}{\PYGZsh{} 输入系统名称(默认为system)}
\PYG{o}{[}按回车使用默认名称或输入之前计算时使用的系统名称\PYG{o}{]}

\PYG{c+c1}{\PYGZsh{} 之后的分析操作与正常计算完成后的分析操作相同}
\end{sphinxVerbatim}


\section{示例:计算蛋白质\sphinxhyphen{}配体结合能}
\label{\detokenize{quick_start:id8}}
\sphinxAtStartPar
下面是一个计算蛋白质\sphinxhyphen{}配体结合能的完整示例:

\begin{sphinxVerbatim}[commandchars=\\\{\}]
\PYG{c+c1}{\PYGZsh{} 启动s\PYGZus{}mmpbsa并加载文件}
s\PYGZus{}mmpbsa
md.tpr
\PYG{l+m}{1}
md\PYGZus{}centered.xtc
\PYG{l+m}{2}
index.ndx
\PYG{l+m}{0}

\PYG{c+c1}{\PYGZsh{} 设置轨迹参数}
\PYG{l+m}{1}
\PYG{l+m}{1}\PYG{+w}{  }\PYG{c+c1}{\PYGZsh{} 假设1是Protein组}
\PYG{l+m}{2}
\PYG{l+m}{13}\PYG{+w}{  }\PYG{c+c1}{\PYGZsh{} 假设13是配体组}
\PYG{l+m}{5}
\PYG{l+m}{1}
\PYG{l+m}{0}

\PYG{c+c1}{\PYGZsh{} 设置MM/PB\PYGZhy{}SA参数(使用默认值)}
\PYG{l+m}{0}

\PYG{c+c1}{\PYGZsh{} 执行计算}
protein\PYGZus{}ligand\PYG{+w}{  }\PYG{c+c1}{\PYGZsh{} 系统名称}

\PYG{c+c1}{\PYGZsh{} 分析结果}
\PYGZhy{}1
\PYG{l+m}{1}
\PYG{l+m}{2}
\PYG{l+m}{3}
\PYG{l+m}{1}
\PYG{l+m}{4}
\PYG{l+m}{0}
\end{sphinxVerbatim}


\section{使用技巧}
\label{\detokenize{quick_start:id9}}\begin{enumerate}
\sphinxsetlistlabels{\arabic}{enumi}{enumii}{}{.}%
\item {} 
\sphinxAtStartPar
\sphinxstylestrong{轨迹准备}:在计算前,建议使用Gromacs的trjconv工具对轨迹进行处理,包括去除PBC、中心化和拟合等操作,以获得更好的计算结果。

\item {} 
\sphinxAtStartPar
\sphinxstylestrong{索引文件}:确保索引文件包含正确的受体和配体组。如果没有现成的索引文件,可以使用Gromacs的make\_ndx工具创建。

\item {} 
\sphinxAtStartPar
\sphinxstylestrong{时间间隔}:对于长时间的MD模拟,可以适当增加时间间隔,减少计算量。通常每1\sphinxhyphen{}2ns分析一次即可获得较好的统计结果。

\item {} 
\sphinxAtStartPar
\sphinxstylestrong{并行计算}:在settings.ini中设置适当的nkernels值,可以利用多核CPU加速计算。

\item {} 
\sphinxAtStartPar
\sphinxstylestrong{结果可视化}:生成的pdb文件可以用PyMOL等软件打开,查看残基结合能的分布情况(通过B因子着色)。

\end{enumerate}


\section{常见问题解答}
\label{\detokenize{quick_start:id10}}
\sphinxAtStartPar
\#\#\# 如何处理大型系统?

\sphinxAtStartPar
对于大型系统,可以尝试以下优化措施:
\begin{itemize}
\item {} 
\sphinxAtStartPar
增加时间间隔,减少分析的帧数

\item {} 
\sphinxAtStartPar
增加nkernels值,利用更多CPU核心

\item {} 
\sphinxAtStartPar
使用较小的截断距离(通过修改r\_cutoff参数)

\end{itemize}

\sphinxAtStartPar
\#\#\# 如何提高计算精度?

\sphinxAtStartPar
提高计算精度的方法包括:
\begin{itemize}
\item {} 
\sphinxAtStartPar
确保轨迹质量良好,已正确处理PBC

\item {} 
\sphinxAtStartPar
增加采样点数,即减小时间间隔

\item {} 
\sphinxAtStartPar
调整PB参数,如网格大小、溶剂介电常数等

\end{itemize}

\sphinxAtStartPar
\#\#\# 计算结果如何解读?

\sphinxAtStartPar
结合自由能的负值越大,表示结合越强。通常,计算结果会给出以下能量项:
\begin{itemize}
\item {} 
\sphinxAtStartPar
ΔG\_bind:总结合自由能

\item {} 
\sphinxAtStartPar
ΔH:焓变

\item {} 
\sphinxAtStartPar
TΔS:熵贡献

\item {} 
\sphinxAtStartPar
ΔE\_vdw:范德华相互作用能

\item {} 
\sphinxAtStartPar
ΔE\_elec:静电相互作用能

\item {} 
\sphinxAtStartPar
ΔG\_polar:极性溶剂化自由能

\item {} 
\sphinxAtStartPar
ΔG\_nonpolar:非极性溶剂化自由能

\end{itemize}

\sphinxAtStartPar
更多详细信息,请参考 {\hyperref[\detokenize{usage::doc}]{\sphinxcrossref{\DUrole{doc}{使用指南}}}} 章节。

\sphinxstepscope


\chapter{使用指南}
\label{\detokenize{usage:id1}}\label{\detokenize{usage::doc}}
\sphinxAtStartPar
本文档详细介绍s\_mmpbsa的各项功能和使用方法,帮助您深入了解并充分利用该工具进行结合自由能计算。


\section{命令行参数}
\label{\detokenize{usage:id2}}
\sphinxAtStartPar
s\_mmpbsa支持以下命令行参数:

\begin{sphinxVerbatim}[commandchars=\\\{\}]
\PYG{c+c1}{\PYGZsh{} 基本用法}
s\PYGZus{}mmpbsa\PYG{+w}{ }\PYG{o}{[}options\PYG{o}{]}\PYG{+w}{ }\PYG{o}{[}tpr文件\PYG{o}{]}

\PYG{c+c1}{\PYGZsh{} 选项}
\PYGZhy{}h,\PYG{+w}{ }\PYGZhy{}\PYGZhy{}help\PYG{+w}{      }显示帮助信息
\PYGZhy{}v,\PYG{+w}{ }\PYGZhy{}\PYGZhy{}version\PYG{+w}{   }显示版本信息
\PYGZhy{}a,\PYG{+w}{ }\PYGZhy{}\PYGZhy{}analysis\PYG{+w}{  }直接进入分析模式
\PYGZhy{}c,\PYG{+w}{ }\PYGZhy{}\PYGZhy{}config\PYG{+w}{    }指定配置文件路径
\end{sphinxVerbatim}


\section{交互式命令行界面}
\label{\detokenize{usage:id3}}
\sphinxAtStartPar
s\_mmpbsa的交互式命令行界面分为以下几个主要部分:
\begin{enumerate}
\sphinxsetlistlabels{\arabic}{enumi}{enumii}{}{.}%
\item {} 
\sphinxAtStartPar
\sphinxstylestrong{文件加载}:加载tpr、xtc和ndx文件

\item {} 
\sphinxAtStartPar
\sphinxstylestrong{轨迹参数设置}:设置受体、配体组和时间间隔等

\item {} 
\sphinxAtStartPar
\sphinxstylestrong{MM/PB\sphinxhyphen{}SA参数设置}:设置计算相关的各种参数

\item {} 
\sphinxAtStartPar
\sphinxstylestrong{计算执行}:执行MM/PB\sphinxhyphen{}SA计算

\item {} 
\sphinxAtStartPar
\sphinxstylestrong{结果分析}:分析和可视化计算结果

\end{enumerate}

\sphinxAtStartPar
下面详细介绍每个部分的操作方法和参数设置。


\section{文件加载}
\label{\detokenize{usage:id4}}
\sphinxAtStartPar
在启动s\_mmpbsa后,首先需要加载必要的输入文件:
\begin{itemize}
\item {} 
\sphinxAtStartPar
\sphinxstylestrong{tpr文件}:Gromacs的拓扑参数文件,包含系统的拓扑信息和原子参数

\item {} 
\sphinxAtStartPar
\sphinxstylestrong{xtc文件}:Gromacs的轨迹文件,包含系统的坐标信息

\item {} 
\sphinxAtStartPar
\sphinxstylestrong{ndx文件}:Gromacs的索引文件,包含系统的分组信息

\end{itemize}

\sphinxAtStartPar
加载文件的示例操作:

\begin{sphinxVerbatim}[commandchars=\\\{\}]
\PYG{c+c1}{\PYGZsh{} 输入tpr文件路径(可以是绝对路径或相对路径)}
md.tpr

\PYG{c+c1}{\PYGZsh{} 选择加载xtc文件}
\PYG{l+m}{1}
md\PYGZus{}centered.xtc\PYG{+w}{  }\PYG{c+c1}{\PYGZsh{} 输入xtc文件路径,按回车使用默认的md.xtc}

\PYG{c+c1}{\PYGZsh{} 选择加载ndx文件}
\PYG{l+m}{2}
index.ndx\PYG{+w}{  }\PYG{c+c1}{\PYGZsh{} 输入ndx文件路径,按回车使用默认的index.ndx}

\PYG{c+c1}{\PYGZsh{} 进入下一步}
\PYG{l+m}{0}
\end{sphinxVerbatim}


\section{轨迹参数设置}
\label{\detokenize{usage:id5}}
\sphinxAtStartPar
加载文件后,需要设置轨迹参数,主要包括:
\begin{itemize}
\item {} 
\sphinxAtStartPar
\sphinxstylestrong{受体组}:选择哪个组作为受体

\item {} 
\sphinxAtStartPar
\sphinxstylestrong{配体组}:选择哪个组作为配体

\item {} 
\sphinxAtStartPar
\sphinxstylestrong{时间间隔}:设置分析的时间间隔

\item {} 
\sphinxAtStartPar
\sphinxstylestrong{跳过的帧数}:设置开始分析前跳过的帧数

\item {} 
\sphinxAtStartPar
\sphinxstylestrong{结束分析的时间点}:设置结束分析的时间点

\end{itemize}

\sphinxAtStartPar
设置轨迹参数的示例操作:

\begin{sphinxVerbatim}[commandchars=\\\{\}]
\PYG{c+c1}{\PYGZsh{} 选择受体组}
\PYG{l+m}{1}
\PYG{l+m}{1}\PYG{+w}{  }\PYG{c+c1}{\PYGZsh{} 输入受体组的编号,例如1表示Protein组}

\PYG{c+c1}{\PYGZsh{} 选择配体组}
\PYG{l+m}{2}
\PYG{l+m}{13}\PYG{+w}{  }\PYG{c+c1}{\PYGZsh{} 输入配体组的编号,例如13表示配体}

\PYG{c+c1}{\PYGZsh{} 设置跳过的帧数}
\PYG{l+m}{3}
\PYG{l+m}{0}\PYG{+w}{  }\PYG{c+c1}{\PYGZsh{} 输入要跳过的帧数,默认为0}

\PYG{c+c1}{\PYGZsh{} 设置结束分析的时间点}
\PYG{l+m}{4}
\PYG{l+m}{0}\PYG{+w}{  }\PYG{c+c1}{\PYGZsh{} 输入结束时间点,0表示分析到轨迹结束}

\PYG{c+c1}{\PYGZsh{} 设置时间间隔(单位:ns)}
\PYG{l+m}{5}
\PYG{l+m}{1}\PYG{+w}{  }\PYG{c+c1}{\PYGZsh{} 输入时间间隔,例如1表示每1ns分析一次}

\PYG{c+c1}{\PYGZsh{} 进入下一步}
\PYG{l+m}{0}
\end{sphinxVerbatim}


\section{MM/PB\sphinxhyphen{}SA参数设置}
\label{\detokenize{usage:mm-pb-sa}}
\sphinxAtStartPar
接下来,设置MM/PB\sphinxhyphen{}SA计算的相关参数。s\_mmpbsa提供了多种参数设置选项:

\begin{sphinxVerbatim}[commandchars=\\\{\}]
\PYG{c+c1}{\PYGZsh{} 显示当前参数设置}
\PYG{l+m}{1}

\PYG{c+c1}{\PYGZsh{} 设置温度(单位:K)}
\PYG{l+m}{2}
\PYG{l+m}{298}.15\PYG{+w}{  }\PYG{c+c1}{\PYGZsh{} 输入温度,默认为298.15K}

\PYG{c+c1}{\PYGZsh{} 设置KCl浓度(单位:mol/L)}
\PYG{l+m}{3}
\PYG{l+m}{0}.15\PYG{+w}{  }\PYG{c+c1}{\PYGZsh{} 输入KCl浓度,默认为0.15mol/L}

\PYG{c+c1}{\PYGZsh{} 设置盐桥搜索距离(单位:Å)}
\PYG{l+m}{4}
\PYG{l+m}{4}.0\PYG{+w}{  }\PYG{c+c1}{\PYGZsh{} 输入盐桥搜索距离,默认为4.0Å}

\PYG{c+c1}{\PYGZsh{} 设置氢键搜索距离(单位:Å)}
\PYG{l+m}{5}
\PYG{l+m}{3}.5\PYG{+w}{  }\PYG{c+c1}{\PYGZsh{} 输入氢键搜索距离,默认为3.5Å}

\PYG{c+c1}{\PYGZsh{} 设置范德华截断距离(单位:Å)}
\PYG{l+m}{6}
\PYG{l+m}{14}.0\PYG{+w}{  }\PYG{c+c1}{\PYGZsh{} 输入范德华截断距离,默认为14.0Å}

\PYG{c+c1}{\PYGZsh{} 设置MM计算的并行核数}
\PYG{l+m}{7}
\PYG{l+m}{4}\PYG{+w}{  }\PYG{c+c1}{\PYGZsh{} 输入并行核数,默认为系统CPU核心数}

\PYG{c+c1}{\PYGZsh{} 设置PB参数}
\PYG{l+m}{8}
\PYG{c+c1}{\PYGZsh{} 输入PB参数设置子菜单(详见下文)}

\PYG{c+c1}{\PYGZsh{} 设置SA参数}
\PYG{l+m}{9}
\PYG{c+c1}{\PYGZsh{} 输入SA参数设置子菜单(详见下文)}

\PYG{c+c1}{\PYGZsh{} 进入下一步}
\PYG{l+m}{0}
\end{sphinxVerbatim}


\subsection{PB参数设置}
\label{\detokenize{usage:pb}}
\sphinxAtStartPar
在PB参数设置子菜单中,可以设置以下参数:

\begin{sphinxVerbatim}[commandchars=\\\{\}]
\PYG{c+c1}{\PYGZsh{} 显示当前PB参数设置}
\PYG{l+m}{1}

\PYG{c+c1}{\PYGZsh{} 设置溶剂介电常数}
\PYG{l+m}{2}
\PYG{l+m}{78}.54\PYG{+w}{  }\PYG{c+c1}{\PYGZsh{} 输入溶剂介电常数,默认为78.54}

\PYG{c+c1}{\PYGZsh{} 设置溶质介电常数}
\PYG{l+m}{3}
\PYG{l+m}{1}.0\PYG{+w}{  }\PYG{c+c1}{\PYGZsh{} 输入溶质介电常数,默认为1.0}

\PYG{c+c1}{\PYGZsh{} 设置网格间距(单位:Å)}
\PYG{l+m}{4}
\PYG{l+m}{0}.5\PYG{+w}{  }\PYG{c+c1}{\PYGZsh{} 输入网格间距,默认为0.5Å}

\PYG{c+c1}{\PYGZsh{} 设置APBS可执行文件路径}
\PYG{l+m}{5}
/usr/local/bin/apbs\PYG{+w}{  }\PYG{c+c1}{\PYGZsh{} 输入APBS可执行文件路径,按回车使用内置路径}

\PYG{c+c1}{\PYGZsh{} 返回上一级菜单}
\PYG{l+m}{0}
\end{sphinxVerbatim}


\subsection{SA参数设置}
\label{\detokenize{usage:sa}}
\sphinxAtStartPar
在SA参数设置子菜单中,可以设置以下参数:

\begin{sphinxVerbatim}[commandchars=\\\{\}]
\PYG{c+c1}{\PYGZsh{} 显示当前SA参数设置}
\PYG{l+m}{1}

\PYG{c+c1}{\PYGZsh{} 设置表面张力(单位:kJ/(mol·Å²))}
\PYG{l+m}{2}
\PYG{l+m}{0}.0379\PYG{+w}{  }\PYG{c+c1}{\PYGZsh{} 输入表面张力,默认为0.0379 kJ/(mol·Å²)}

\PYG{c+c1}{\PYGZsh{} 设置非极性溶剂化参数(单位:kJ/(mol·Å³))}
\PYG{l+m}{3}
\PYG{l+m}{0}.0\PYG{+w}{  }\PYG{c+c1}{\PYGZsh{} 输入非极性溶剂化参数,默认为0.0 kJ/(mol·Å³)}

\PYG{c+c1}{\PYGZsh{} 设置SAS计算方法(0:Shrake\PYGZhy{}Rupley,1:MSMS)}
\PYG{l+m}{4}
\PYG{l+m}{0}\PYG{+w}{  }\PYG{c+c1}{\PYGZsh{} 输入SAS计算方法,默认为0(Shrake\PYGZhy{}Rupley)}

\PYG{c+c1}{\PYGZsh{} 返回上一级菜单}
\PYG{l+m}{0}
\end{sphinxVerbatim}


\section{执行计算}
\label{\detokenize{usage:id6}}
\sphinxAtStartPar
设置完成后,开始执行MM/PB\sphinxhyphen{}SA计算。在计算前,需要输入系统名称:

\begin{sphinxVerbatim}[commandchars=\\\{\}]
\PYG{c+c1}{\PYGZsh{} 输入系统名称}
protein\PYGZus{}ligand\PYG{+w}{  }\PYG{c+c1}{\PYGZsh{} 输入系统名称,默认为system}

\PYG{c+c1}{\PYGZsh{} 计算过程中会显示进度条和当前能量值}
\PYG{c+c1}{\PYGZsh{} 计算完成后,会自动进入分析模式}
\end{sphinxVerbatim}


\section{结果分析}
\label{\detokenize{usage:id7}}
\sphinxAtStartPar
计算完成后,可以进行结果分析。s\_mmpbsa提供了多种分析选项:

\begin{sphinxVerbatim}[commandchars=\\\{\}]
\PYG{c+c1}{\PYGZsh{} 生成包含残基结合能信息的pdb文件}
\PYGZhy{}1
\PYG{l+m}{0}\PYG{+w}{  }\PYG{c+c1}{\PYGZsh{} 输入时间点,0表示平均值}

\PYG{c+c1}{\PYGZsh{} 查看结果摘要}
\PYG{l+m}{1}

\PYG{c+c1}{\PYGZsh{} 输出能量随时间变化的数据}
\PYG{l+m}{2}

\PYG{c+c1}{\PYGZsh{} 输出特定时间点的残基结合能}
\PYG{l+m}{3}
\PYG{l+m}{0}\PYG{+w}{  }\PYG{c+c1}{\PYGZsh{} 输入时间点,0表示平均值}
\PYG{l+m}{1}\PYG{+w}{  }\PYG{c+c1}{\PYGZsh{} 选择输出3Å范围内的残基(0:全部残基,1:3Å范围,2:5Å范围,3:10Å范围)}

\PYG{c+c1}{\PYGZsh{} 输出配体原子的结合能}
\PYG{l+m}{4}

\PYG{c+c1}{\PYGZsh{} 输出氢键和盐桥信息}
\PYG{l+m}{5}

\PYG{c+c1}{\PYGZsh{} 输出相互作用能矩阵}
\PYG{l+m}{6}

\PYG{c+c1}{\PYGZsh{} 退出程序}
\PYG{l+m}{0}
\end{sphinxVerbatim}


\section{使用分析模式}
\label{\detokenize{usage:id8}}
\sphinxAtStartPar
s\_mmpbsa还提供了专门的分析模式,可以对已计算的结果进行重新分析,而无需重新计算:

\begin{sphinxVerbatim}[commandchars=\\\{\}]
\PYG{c+c1}{\PYGZsh{} 启动分析模式}
s\PYGZus{}mmpbsa\PYG{+w}{ }\PYGZhy{}a\PYG{+w}{  }\PYG{c+c1}{\PYGZsh{} 使用命令行参数直接进入分析模式}
\PYG{c+c1}{\PYGZsh{} 或者}
s\PYGZus{}mmpbsa
a\PYG{+w}{  }\PYG{c+c1}{\PYGZsh{} 在交互式界面中输入a进入分析模式}

\PYG{c+c1}{\PYGZsh{} 输入工作目录路径}
./results\PYG{+w}{  }\PYG{c+c1}{\PYGZsh{} 输入包含.sm结果文件的目录,默认为当前目录}

\PYG{c+c1}{\PYGZsh{} 输入温度(单位:K)}
\PYG{l+m}{298}.15\PYG{+w}{  }\PYG{c+c1}{\PYGZsh{} 输入温度,默认为298.15K}

\PYG{c+c1}{\PYGZsh{} 输入系统名称}
protein\PYGZus{}ligand\PYG{+w}{  }\PYG{c+c1}{\PYGZsh{} 输入之前计算时使用的系统名称,默认为system}

\PYG{c+c1}{\PYGZsh{} 之后的分析操作与正常计算完成后的分析操作相同}
\end{sphinxVerbatim}


\section{丙氨酸扫描}
\label{\detokenize{usage:id9}}
\sphinxAtStartPar
s\_mmpbsa还支持丙氨酸扫描功能,可以系统性地将蛋白质残基突变为丙氨酸,并计算突变前后的结合自由能变化。

\sphinxAtStartPar
进行丙氨酸扫描的步骤如下:

\begin{sphinxVerbatim}[commandchars=\\\{\}]
\PYG{c+c1}{\PYGZsh{} 准备工作目录}
mkdir\PYG{+w}{ }\PYGZhy{}p\PYG{+w}{ }ala\PYGZus{}scan
\PYG{n+nb}{cd}\PYG{+w}{ }ala\PYGZus{}scan

\PYG{c+c1}{\PYGZsh{} 复制必要的输入文件}
cp\PYG{+w}{ }../md.tpr\PYG{+w}{ }../md\PYGZus{}centered.xtc\PYG{+w}{ }../index.ndx\PYG{+w}{ }.

\PYG{c+c1}{\PYGZsh{} 执行丙氨酸扫描}
s\PYGZus{}mmpbsa\PYG{+w}{ }md.tpr
\PYG{l+m}{1}
md\PYGZus{}centered.xtc
\PYG{l+m}{2}
index.ndx
\PYG{l+m}{0}
\PYG{l+m}{1}
\PYG{l+m}{1}\PYG{+w}{  }\PYG{c+c1}{\PYGZsh{} 选择受体组(蛋白质)}
\PYG{l+m}{2}
\PYG{l+m}{13}\PYG{+w}{  }\PYG{c+c1}{\PYGZsh{} 选择配体组}
\PYG{l+m}{5}
\PYG{l+m}{1}
\PYG{l+m}{0}
\PYG{l+m}{0}
protein\PYGZus{}ligand\PYG{+w}{  }\PYG{c+c1}{\PYGZsh{} 系统名称}
\PYGZhy{}1\PYG{+w}{  }\PYG{c+c1}{\PYGZsh{} 生成pdb文件(可选)}
\PYG{l+m}{0}\PYG{+w}{  }\PYG{c+c1}{\PYGZsh{} 退出分析}
a\PYG{+w}{  }\PYG{c+c1}{\PYGZsh{} 进入分析模式}
.\PYG{+w}{  }\PYG{c+c1}{\PYGZsh{} 使用当前目录}
\PYG{l+m}{298}.15\PYG{+w}{  }\PYG{c+c1}{\PYGZsh{} 温度}
protein\PYGZus{}ligand\PYG{+w}{  }\PYG{c+c1}{\PYGZsh{} 系统名称}
\PYG{l+m}{0}\PYG{+w}{  }\PYG{c+c1}{\PYGZsh{} 退出分析}

\PYG{c+c1}{\PYGZsh{} 现在可以查看丙氨酸扫描的结果}
\PYG{c+c1}{\PYGZsh{} 结果文件为protein\PYGZus{}ligand\PYGZus{}sm\PYGZus{}ala\PYGZus{}results.csv}
\end{sphinxVerbatim}


\section{对接重打分}
\label{\detokenize{usage:id10}}
\sphinxAtStartPar
s\_mmpbsa也支持对接重打分功能,可以对分子对接的结果进行MM/PB\sphinxhyphen{}SA计算,评估不同构象的结合强度。

\sphinxAtStartPar
进行对接重打分的步骤如下:

\begin{sphinxVerbatim}[commandchars=\\\{\}]
\PYG{c+c1}{\PYGZsh{} 准备工作目录}
mkdir\PYG{+w}{ }\PYGZhy{}p\PYG{+w}{ }docking\PYGZus{}rescoring
\PYG{n+nb}{cd}\PYG{+w}{ }docking\PYGZus{}rescoring

\PYG{c+c1}{\PYGZsh{} 确保有对接后的pdb文件(例如docked.pdb)}

\PYG{c+c1}{\PYGZsh{} 执行对接重打分}
s\PYGZus{}mmpbsa\PYG{+w}{ }md.tpr
\PYG{l+m}{1}
docked.pdb\PYG{+w}{  }\PYG{c+c1}{\PYGZsh{} 输入对接后的pdb文件代替轨迹文件}
\PYG{l+m}{2}
index.ndx
\PYG{l+m}{0}
\PYG{l+m}{1}
\PYG{l+m}{1}\PYG{+w}{  }\PYG{c+c1}{\PYGZsh{} 选择受体组(蛋白质)}
\PYG{l+m}{2}
\PYG{l+m}{13}\PYG{+w}{  }\PYG{c+c1}{\PYGZsh{} 选择配体组}
\PYG{l+m}{5}
\PYG{l+m}{0}.1\PYG{+w}{  }\PYG{c+c1}{\PYGZsh{} 时间间隔(对于pdb文件,此参数不影响结果)}
\PYG{l+m}{0}
\PYG{l+m}{0}
docking\PYGZus{}rescore\PYG{+w}{  }\PYG{c+c1}{\PYGZsh{} 系统名称}
\PYGZhy{}1\PYG{+w}{  }\PYG{c+c1}{\PYGZsh{} 生成pdb文件(可选)}
\PYG{l+m}{0}\PYG{+w}{  }\PYG{c+c1}{\PYGZsh{} 退出分析}
\end{sphinxVerbatim}


\section{计算结果解读}
\label{\detokenize{usage:id11}}
\sphinxAtStartPar
s\_mmpbsa的计算结果主要包括以下能量项:
\begin{itemize}
\item {} 
\sphinxAtStartPar
\sphinxstylestrong{ΔG\_bind}:总结合自由能

\item {} 
\sphinxAtStartPar
\sphinxstylestrong{ΔH}:焓变

\item {} 
\sphinxAtStartPar
\sphinxstylestrong{TΔS}:熵贡献(注意:s\_mmpbsa目前不直接计算熵,此值通常设为0或通过其他方法估算)

\item {} 
\sphinxAtStartPar
\sphinxstylestrong{ΔE\_vdw}:范德华相互作用能

\item {} 
\sphinxAtStartPar
\sphinxstylestrong{ΔE\_elec}:静电相互作用能

\item {} 
\sphinxAtStartPar
\sphinxstylestrong{ΔG\_polar}:极性溶剂化自由能

\item {} 
\sphinxAtStartPar
\sphinxstylestrong{ΔG\_nonpolar}:非极性溶剂化自由能

\end{itemize}

\sphinxAtStartPar
结合自由能的负值越大,表示结合越强。通常,ΔG\_bind < \sphinxhyphen{}10 kJ/mol表示较强的结合。


\section{注意事项}
\label{\detokenize{usage:id12}}
\sphinxAtStartPar
在使用s\_mmpbsa时,需要注意以下几点:
\begin{enumerate}
\sphinxsetlistlabels{\arabic}{enumi}{enumii}{}{.}%
\item {} 
\sphinxAtStartPar
\sphinxstylestrong{轨迹质量}:确保轨迹质量良好,已正确处理PBC、中心化和拟合等操作。

\item {} 
\sphinxAtStartPar
\sphinxstylestrong{索引文件}:确保索引文件包含正确的受体和配体组。

\item {} 
\sphinxAtStartPar
\sphinxstylestrong{参数选择}:对于不同的系统,可能需要调整参数以获得更准确的结果。

\item {} 
\sphinxAtStartPar
\sphinxstylestrong{并行计算}:在settings.ini中设置适当的nkernels值,可以利用多核CPU加速计算。

\item {} 
\sphinxAtStartPar
\sphinxstylestrong{结果验证}:建议与实验数据或其他计算方法的结果进行比较,验证计算结果的可靠性。

\end{enumerate}


\section{更多信息}
\label{\detokenize{usage:id13}}\begin{itemize}
\item {} 
\sphinxAtStartPar
{\hyperref[\detokenize{quick_start::doc}]{\sphinxcrossref{\DUrole{doc}{快速入门}}}}:快速入门指南

\item {} 
\sphinxAtStartPar
{\hyperref[\detokenize{installation::doc}]{\sphinxcrossref{\DUrole{doc}{安装}}}}:安装说明

\item {} 
\sphinxAtStartPar
\DUrole{xref}{\DUrole{std}{\DUrole{std-doc}{api}}}:API文档(适用于开发者)

\item {} 
\sphinxAtStartPar
{\hyperref[\detokenize{faq::doc}]{\sphinxcrossref{\DUrole{doc}{常见问题解答}}}}:常见问题解答

\end{itemize}

\sphinxstepscope


\chapter{常见问题解答}
\label{\detokenize{faq:id1}}\label{\detokenize{faq::doc}}
\sphinxAtStartPar
本文档解答用户在使用s\_mmpbsa过程中可能遇到的常见问题,帮助您快速解决使用过程中遇到的困难。


\section{安装问题}
\label{\detokenize{faq:id2}}
\sphinxAtStartPar
Q: 安装s\_mmpbsa后,运行时提示找不到Gromacs,应该如何解决?

\sphinxAtStartPar
A: 这个问题通常是因为系统中没有正确安装Gromacs或者Gromacs的可执行文件没有添加到系统路径中。您可以通过以下方法解决:
\begin{enumerate}
\sphinxsetlistlabels{\arabic}{enumi}{enumii}{}{.}%
\item {} 
\sphinxAtStartPar
确保已经正确安装了Gromacs(建议版本5.1或更高)

\item {} 
\sphinxAtStartPar
将Gromacs的bin目录添加到系统的PATH环境变量中

\item {} 
\sphinxAtStartPar
或者在s\_mmpbsa的settings.ini文件中手动指定Gromacs的路径

\end{enumerate}

\sphinxAtStartPar
Windows系统的settings.ini文件通常位于s\_mmpbsa的安装目录下,Linux系统的settings.ini文件通常位于\textasciitilde{}/.config/s\_mmpbsa/目录下。

\sphinxAtStartPar
Q: 运行s\_mmpbsa时提示缺少APBS,应该如何安装APBS?

\sphinxAtStartPar
A: APBS(Adaptive Poisson\sphinxhyphen{}Boltzmann Solver)是计算PB能量所必需的外部程序。您可以通过以下方法安装APBS:

\sphinxAtStartPar
\sphinxstylestrong{Linux系统}:

\begin{sphinxVerbatim}[commandchars=\\\{\}]
\PYG{c+c1}{\PYGZsh{} Ubuntu/Debian}
sudo\PYG{+w}{ }apt\PYGZhy{}get\PYG{+w}{ }install\PYG{+w}{ }apbs

\PYG{c+c1}{\PYGZsh{} CentOS/RHEL}
sudo\PYG{+w}{ }yum\PYG{+w}{ }install\PYG{+w}{ }apbs
\end{sphinxVerbatim}

\sphinxAtStartPar
\sphinxstylestrong{Windows系统}:
\begin{enumerate}
\sphinxsetlistlabels{\arabic}{enumi}{enumii}{}{.}%
\item {} 
\sphinxAtStartPar
从APBS官网(\sphinxurl{https://apbs-pdb2pqr.readthedocs.io/en/latest/downloads.html})下载Windows安装包

\item {} 
\sphinxAtStartPar
安装APBS并将其添加到系统PATH环境变量中

\end{enumerate}

\sphinxAtStartPar
安装完成后,您可能需要在s\_mmpbsa的settings.ini文件中手动指定APBS的路径。

\sphinxAtStartPar
Q: 在Windows系统上,运行s\_mmpbsa时出现"无法找到入口点"的错误,应该如何解决?

\sphinxAtStartPar
A: 这个问题通常是因为缺少必要的Visual C++ Redistributable运行库。您可以从Microsoft官网下载并安装Visual C++ Redistributable for Visual Studio 2019或更高版本,以解决这个问题。


\section{使用问题}
\label{\detokenize{faq:id3}}
\sphinxAtStartPar
Q: 如何准备s\_mmpbsa的输入文件?

\sphinxAtStartPar
A: s\_mmpbsa需要以下输入文件:
\begin{enumerate}
\sphinxsetlistlabels{\arabic}{enumi}{enumii}{}{.}%
\item {} 
\sphinxAtStartPar
\sphinxstylestrong{tpr文件}:使用Gromacs的grompp命令生成

\item {} 
\sphinxAtStartPar
\sphinxstylestrong{xtc文件}:使用Gromacs的mdrun命令生成的轨迹文件

\item {} 
\sphinxAtStartPar
\sphinxstylestrong{ndx文件}:包含受体和配体组的索引文件,可以使用Gromacs的make\_ndx命令创建

\end{enumerate}

\sphinxAtStartPar
为了获得更好的计算结果,建议在使用s\_mmpbsa前对轨迹文件进行预处理,包括去除PBC、中心化和拟合等操作。您可以使用Gromacs的trjconv命令进行这些操作:

\begin{sphinxVerbatim}[commandchars=\\\{\}]
gmx\PYG{+w}{ }trjconv\PYG{+w}{ }\PYGZhy{}s\PYG{+w}{ }md.tpr\PYG{+w}{ }\PYGZhy{}f\PYG{+w}{ }md.xtc\PYG{+w}{ }\PYGZhy{}o\PYG{+w}{ }md\PYGZus{}centered.xtc\PYG{+w}{ }\PYGZhy{}pbc\PYG{+w}{ }mol\PYG{+w}{ }\PYGZhy{}center\PYG{+w}{ }\PYGZhy{}ur\PYG{+w}{ }compact
\end{sphinxVerbatim}

\sphinxAtStartPar
Q: 如何选择合适的时间间隔?

\sphinxAtStartPar
A: 时间间隔的选择取决于您的模拟长度和计算资源。对于较短的模拟(如100 ns以下),可以选择较小的时间间隔(如0.5\sphinxhyphen{}1 ns);对于较长的模拟(如100 ns以上),可以选择较大的时间间隔(如1\sphinxhyphen{}2 ns)。

\sphinxAtStartPar
一般来说,建议至少分析10\sphinxhyphen{}20个时间点,以获得较好的统计结果。时间间隔太小会增加计算量,时间间隔太大则可能丢失重要的动力学信息。

\sphinxAtStartPar
Q: 如何提高计算速度?

\sphinxAtStartPar
A: 您可以通过以下方法提高s\_mmpbsa的计算速度:
\begin{enumerate}
\sphinxsetlistlabels{\arabic}{enumi}{enumii}{}{.}%
\item {} 
\sphinxAtStartPar
在MM/PB\sphinxhyphen{}SA参数设置中增加并行核数(nkernels)

\item {} 
\sphinxAtStartPar
增大时间间隔,减少分析的帧数

\item {} 
\sphinxAtStartPar
增大范德华截断距离(r\_cutoff),减少计算的相互作用对数量

\item {} 
\sphinxAtStartPar
使用较小的网格间距(grid\_spacing)进行PB计算

\end{enumerate}

\sphinxAtStartPar
Q: 如何解释计算结果?

\sphinxAtStartPar
A: s\_mmpbsa的计算结果主要包括以下能量项:
\begin{itemize}
\item {} 
\sphinxAtStartPar
\sphinxstylestrong{ΔG\_bind}:总结合自由能,负值越大表示结合越强

\item {} 
\sphinxAtStartPar
\sphinxstylestrong{ΔE\_vdw}:范德华相互作用能,通常为负值,表示吸引力

\item {} 
\sphinxAtStartPar
\sphinxstylestrong{ΔE\_elec}:静电相互作用能,可能为正值或负值

\item {} 
\sphinxAtStartPar
\sphinxstylestrong{ΔG\_polar}:极性溶剂化自由能,通常为正值,表示溶剂化 penalty

\item {} 
\sphinxAtStartPar
\sphinxstylestrong{ΔG\_nonpolar}:非极性溶剂化自由能,通常为负值,表示疏水效应

\end{itemize}

\sphinxAtStartPar
结合自由能的计算值应该与实验值进行比较,以验证计算结果的可靠性。一般来说,计算值与实验值的误差在2\sphinxhyphen{}3 kcal/mol(约8\sphinxhyphen{}12 kJ/mol)范围内是可以接受的。


\section{技术问题}
\label{\detokenize{faq:id4}}
\sphinxAtStartPar
Q: 计算过程中出现"内存不足"的错误,应该如何解决?

\sphinxAtStartPar
A: 内存不足的问题通常出现在处理大型系统时。您可以通过以下方法解决:
\begin{enumerate}
\sphinxsetlistlabels{\arabic}{enumi}{enumii}{}{.}%
\item {} 
\sphinxAtStartPar
减小时间间隔,减少同时加载到内存中的帧数

\item {} 
\sphinxAtStartPar
增加系统的物理内存或虚拟内存

\item {} 
\sphinxAtStartPar
分割轨迹文件,分批次进行计算

\item {} 
\sphinxAtStartPar
对大型系统,考虑使用较小的截断距离

\end{enumerate}

\sphinxAtStartPar
Q: 如何处理带有金属离子的系统?

\sphinxAtStartPar
A: 对于带有金属离子的系统,您需要特别注意以下几点:
\begin{enumerate}
\sphinxsetlistlabels{\arabic}{enumi}{enumii}{}{.}%
\item {} 
\sphinxAtStartPar
确保金属离子的力场参数正确

\item {} 
\sphinxAtStartPar
在计算PB能量时,可能需要调整金属离子的电荷和半径参数

\item {} 
\sphinxAtStartPar
考虑金属离子对溶剂化能的特殊影响

\end{enumerate}

\sphinxAtStartPar
Q: 如何在丙氨酸扫描中排除某些残基?

\sphinxAtStartPar
A: 目前,s\_mmpbsa的丙氨酸扫描功能会自动扫描受体组中的所有残基(除了甘氨酸和丙氨酸本身)。如果您想排除某些残基,可以在进行丙氨酸扫描前修改ndx文件,创建一个只包含您感兴趣的残基的新组。

\sphinxAtStartPar
Q: s\_mmpbsa是否支持GPU加速?

\sphinxAtStartPar
A: 目前,s\_mmpbsa的主要计算部分(如MM能量计算)还不支持GPU加速,但PB计算依赖的APBS程序支持GPU加速。如果您的系统中安装了支持GPU的APBS版本,s\_mmpbsa会自动利用APBS的GPU加速功能。


\section{结果分析问题}
\label{\detokenize{faq:id5}}
\sphinxAtStartPar
Q: 如何将s\_mmpbsa的结果与其他软件的结果进行比较?

\sphinxAtStartPar
A: 将s\_mmpbsa的结果与其他软件(如g\_mmpbsa、AMBER等)的结果进行比较时,需要注意以下几点:
\begin{enumerate}
\sphinxsetlistlabels{\arabic}{enumi}{enumii}{}{.}%
\item {} 
\sphinxAtStartPar
确保使用相同的力场参数和拓扑文件

\item {} 
\sphinxAtStartPar
确保使用相同的轨迹文件和时间间隔

\item {} 
\sphinxAtStartPar
确保使用相同的溶剂化模型参数(如介电常数、盐浓度等)

\item {} 
\sphinxAtStartPar
注意不同软件对能量单位的处理(有些使用kcal/mol,有些使用kJ/mol)

\end{enumerate}

\sphinxAtStartPar
Q: 如何将s\_mmpbsa的结果可视化?

\sphinxAtStartPar
A: s\_mmpbsa提供了以下几种可视化结果的方法:
\begin{enumerate}
\sphinxsetlistlabels{\arabic}{enumi}{enumii}{}{.}%
\item {} 
\sphinxAtStartPar
生成包含残基结合能信息的pdb文件,可以用PyMOL等软件打开并通过B因子着色

\item {} 
\sphinxAtStartPar
输出能量随时间变化的数据,可以用Excel、Origin等软件绘制图表

\item {} 
\sphinxAtStartPar
输出残基结合能数据,可以用热图等方式可视化

\end{enumerate}

\sphinxAtStartPar
Q: 残基结合能的计算结果与预期不符,应该如何处理?

\sphinxAtStartPar
A: 如果残基结合能的计算结果与预期不符,您可以考虑以下几点:
\begin{enumerate}
\sphinxsetlistlabels{\arabic}{enumi}{enumii}{}{.}%
\item {} 
\sphinxAtStartPar
检查输入文件的质量,确保轨迹文件已正确处理PBC

\item {} 
\sphinxAtStartPar
检查索引文件,确保受体和配体组的选择正确

\item {} 
\sphinxAtStartPar
调整MM/PB\sphinxhyphen{}SA参数,如截断距离、网格间距等

\item {} 
\sphinxAtStartPar
考虑使用不同的溶剂化模型参数

\item {} 
\sphinxAtStartPar
增加采样点数,提高统计精度

\end{enumerate}


\section{其他问题}
\label{\detokenize{faq:id6}}
\sphinxAtStartPar
Q: s\_mmpbsa是否支持其他分子动力学软件的轨迹文件?

\sphinxAtStartPar
A: 目前,s\_mmpbsa主要支持Gromacs的轨迹文件(xtc格式)。如果您想使用其他分子动力学软件(如AMBER、NAMD等)的轨迹文件,需要先将其转换为xtc格式或pdb格式。

\sphinxAtStartPar
Q: 如何获取s\_mmpbsa的最新版本?

\sphinxAtStartPar
A: 您可以通过以下方式获取s\_mmpbsa的最新版本:
\begin{enumerate}
\sphinxsetlistlabels{\arabic}{enumi}{enumii}{}{.}%
\item {} 
\sphinxAtStartPar
从GitHub仓库(\sphinxurl{https://github.com/your\_username/s\_mmpbsa})下载源码并自行编译

\item {} 
\sphinxAtStartPar
从项目官网下载预编译的可执行文件

\end{enumerate}

\sphinxAtStartPar
Q: 如何报告bug或提出新功能建议?

\sphinxAtStartPar
A: 您可以通过以下方式报告bug或提出新功能建议:
\begin{enumerate}
\sphinxsetlistlabels{\arabic}{enumi}{enumii}{}{.}%
\item {} 
\sphinxAtStartPar
在GitHub仓库的Issues页面提交bug报告或功能请求

\item {} 
\sphinxAtStartPar
发送电子邮件给开发者(\sphinxhref{mailto:email@example.com}{email@example.com})

\item {} 
\sphinxAtStartPar
加入QQ群(群号:123456789)进行讨论

\end{enumerate}

\sphinxAtStartPar
Q: 如何引用s\_mmpbsa?

\sphinxAtStartPar
A: 如果您在学术研究中使用了s\_mmpbsa,请按照以下格式引用:

\sphinxAtStartPar
作者姓名. s\_mmpbsa: Gromacs轨迹的MM/PB\sphinxhyphen{}SA结合自由能计算工具. 版本号. GitHub. \sphinxurl{https://github.com/your\_username/s\_mmpbsa}


\section{更多信息}
\label{\detokenize{faq:id7}}\begin{itemize}
\item {} 
\sphinxAtStartPar
{\hyperref[\detokenize{usage::doc}]{\sphinxcrossref{\DUrole{doc}{使用指南}}}}:使用指南

\item {} 
\sphinxAtStartPar
{\hyperref[\detokenize{installation::doc}]{\sphinxcrossref{\DUrole{doc}{安装}}}}:安装说明

\item {} 
\sphinxAtStartPar
\DUrole{xref}{\DUrole{std}{\DUrole{std-doc}{api}}}:API文档

\item {} 
\sphinxAtStartPar
{\hyperref[\detokenize{quick_start::doc}]{\sphinxcrossref{\DUrole{doc}{快速入门}}}}:快速入门指南

\end{itemize}


\chapter{简介}
\label{\detokenize{index:id1}}
\sphinxAtStartPar
\sphinxstylestrong{s\_mmpbsa} 提供了一个便捷的界面(类似 \sphinxtitleref{Multiwfn})来计算Gromacs轨迹的结合自由能。相比于其他同类工具,它具有安装简便、运行高效、跨平台等优势,且支持多种高级功能,如电荷筛选效应和构象熵的计算。


\chapter{主要功能}
\label{\detokenize{index:id2}}\begin{itemize}
\item {} 
\sphinxAtStartPar
\sphinxstylestrong{MD模拟结合能计算}:从分子动力学模拟结果计算生物分子间的结合自由能

\item {} 
\sphinxAtStartPar
\sphinxstylestrong{分子对接结果重打分}:为分子对接结果提供更准确的结合能预测

\item {} 
\sphinxAtStartPar
\sphinxstylestrong{蛋白质\sphinxhyphen{}配体复合物丙氨酸扫描}:分析蛋白质中关键残基对结合的贡献

\end{itemize}


\chapter{特点}
\label{\detokenize{index:id3}}\begin{itemize}
\item {} 
\sphinxAtStartPar
开源免费,遵循LGPL许可证

\item {} 
\sphinxAtStartPar
环境依赖少,Linux系统仅需Gromacs程序,绘图功能需Python环境

\item {} 
\sphinxAtStartPar
使用Rust语言开发,性能优异

\item {} 
\sphinxAtStartPar
交互式操作,无需编写参数文件

\item {} 
\sphinxAtStartPar
考虑电荷筛选效应,如文献 {[}J. Chem. Inf. Model. 2021, 61, 2454{]} 所述

\item {} 
\sphinxAtStartPar
考虑构象熵,如文献 {[}J. Chem. Phys. 2017, 146, 124124{]} 所述

\item {} 
\sphinxAtStartPar
可存储分析结果,便于进一步的可重复分析

\end{itemize}


\chapter{开始使用}
\label{\detokenize{index:id4}}
\sphinxAtStartPar
请参考 {\hyperref[\detokenize{installation::doc}]{\sphinxcrossref{\DUrole{doc}{安装}}}} 章节安装s\_mmpbsa,然后查看 {\hyperref[\detokenize{quick_start::doc}]{\sphinxcrossref{\DUrole{doc}{快速入门}}}} 章节了解基本使用流程。

\sphinxAtStartPar
对于详细的使用说明,请参阅 {\hyperref[\detokenize{usage::doc}]{\sphinxcrossref{\DUrole{doc}{使用指南}}}} 章节,其中包含了各种功能的使用方法和实例。


\chapter{获取帮助}
\label{\detokenize{index:id5}}
\sphinxAtStartPar
如果您在使用过程中遇到任何问题,或者有任何改进建议,请联系开发者或加入QQ群:
\begin{itemize}
\item {} 
\sphinxAtStartPar
\sphinxstylestrong{开发者}:张嘉兴博士 (\sphinxhref{mailto:zhangjiaxing7137@tju.edu.cn}{zhangjiaxing7137@tju.edu.cn}, 天津大学)

\item {} 
\sphinxAtStartPar
\sphinxstylestrong{QQ群}:864191465

\end{itemize}


\chapter{引用}
\label{\detokenize{index:id6}}
\sphinxAtStartPar
如果您在研究工作中使用了s\_mmpbsa,请按照以下格式引用:

\begin{sphinxVerbatim}[commandchars=\\\{\}]
Jiaxing Zhang, s\PYGZus{}mmpbsa, Version [your version], https://github.com/supernova4869/s\PYGZus{}mmpbsa (accessed on yy\PYGZhy{}mm\PYGZhy{}dd)
\end{sphinxVerbatim}

\begin{sphinxadmonition}{note}{备注:}
\sphinxAtStartPar
当s\_mmpbsa的详细论文发表后,请引用相应的论文而非此处的网页。
\end{sphinxadmonition}


\chapter{索引与表格}
\label{\detokenize{index:id7}}\begin{itemize}
\item {} 
\sphinxAtStartPar
\DUrole{xref}{\DUrole{std}{\DUrole{std-ref}{genindex}}}

\item {} 
\sphinxAtStartPar
\DUrole{xref}{\DUrole{std}{\DUrole{std-ref}{modindex}}}

\item {} 
\sphinxAtStartPar
\DUrole{xref}{\DUrole{std}{\DUrole{std-ref}{search}}}

\end{itemize}



\renewcommand{\indexname}{索引}
\printindex
\end{document}